\documentclass{article}

% if you need to pass options to natbib, use, e.g.:
% \PassOptionsToPackage{numbers, compress}{natbib}
% before loading nips_2016
%
% to avoid loading the natbib package, add option nonatbib:
% \usepackage[nonatbib]{nips_2016}

\usepackage[final]{nips_2016_customized}

% to compile a camera-ready version, add the [final] option, e.g.:
% \usepackage[final]{nips_2016}

\usepackage[utf8]{inputenc} % allow utf-8 input
\usepackage[T1]{fontenc}    % use 8-bit T1 fonts
\usepackage{hyperref}       % hyperlinks
\usepackage{url}            % simple URL typesetting
\usepackage{booktabs}       % professional-quality tables
\usepackage{amsfonts}       % blackboard math symbols
\usepackage{nicefrac}       % compact symbols for 1/2, etc.
\usepackage{microtype}      % microtypography

\usepackage{minted}

\newcommand{\NAME}{NAME\_TBD}
\newcommand{\TODO}[1]{}
\usepackage{color}
\renewcommand{\TODO}[1]{{\color{red} TODO: {#1}}}

\title{\NAME - A Domain Specific Language for Implementing Finite State Machines in Hardware}

% The \author macro works with any number of authors. There are two
% commands used to separate the names and addresses of multiple
% authors: \And and \AND.
%
% Using \And between authors leaves it to LaTeX to determine where to
% break the lines. Using \AND forces a line break at that point. So,
% if LaTeX puts 3 of 4 authors names on the first line, and the last
% on the second line, try using \AND instead of \And before the third
% author name.

\author{
  Leonard Truong\\%\thanks{http://truong.io} \\
  \texttt{lenny@stanford.edu} \\
  %% examples of more authors
  \And
  James Hegarty \\
  \texttt{jhegarty@stanford.edu} \\
  \AND
  Ross Daly \\
  \texttt{ross.daly@stanford.edu} \\
  \And
  Pat Hanrahan \\
  \texttt{hanrahan@cs.stanford.edu} \\
  %% \AND
  %% Coauthor \\
  %% Affiliation \\
  %% Address \\
  %% \texttt{email} \\
  %% \And
  %% Coauthor \\
  %% Affiliation \\
  %% Address \\
  %% \texttt{email} \\
  %% \And
  %% Coauthor \\
  %% Affiliation \\
  %% Address \\
  %% \texttt{email} \\
}

\begin{document}
% \nipsfinalcopy is no longer used

\maketitle

\begin{abstract}
\end{abstract}

\section{Introduction}
We propose a domain specific language for implementing finite state
machines in hardware. The language is embedded in C to provide seamless
interoperability with existing codebases in any language \TODO{Well any
language with a good FFI could make this argument, probably need to
reword this to avoid religious arguments}. The user is provided with a
set of operators, types, and built-in functions that are described in
detail in Section~\ref{sec:language_specification}. The key language
extension is the introduction of the \texttt{yield} keyword. This
allows users to implement finite state machines using the coroutine programming
model (described in detail in Section~\ref{sec:fsms_as_coroutines}). The
compiler, described in Section~\ref{sec:compiler}, analyzes the program to
construct an abstract representation of the state machine, which is then
lowered into CoreIR.

A side-goal of this project is to construct a lightweight toolkit for
embedding DSLs in C. While embedding a language in C is likely to
introduce some headaches for the toolkit (i.e. we can't just hook into
the built-in parser like in Lua or Python), we see this as a
future-proof host language choice. \TODO{That is, C isn't going
anywhere, so at least the host language won't become obsolete.}

\subsection{Motivation}
A hardware implementation of a finite state machine would traditionally
be written in a hardware description language (HDL) such as Verilog. In
practice, this partially reduces to a giant \texttt{case} statement
that enumerates all the possible state transitions. This approach
suffers from verbosity and is prone to errors. \TODO{Probably good to have James H. (and Ross?) to write something here since they've actually dealt with this}

An alternative approach would be to write the state machine in C and
use a high-level synthesis tool (HLS) to automatically construct a
hardware implementation. \TODO{Argument here is basically general
purpose versus DSL compilers, that is, HLS=general purpose compiler} 
Being able to describe the program in a higher-level, imperative
language is convenient for the programmer, but makes the job difficult
for the compiler. For example, typical HLS systems decouple the
behavior of the program from clock-level timing, which can make it
difficult to understand the semantics of how operations are executed in
hardware (sequential or parallel). This creates issues in designing and debugging implementations.

\subsection{Background}
\subsubsection{Automatons and Finite State Machines}
\emph{Automatons} are abstract models of machines that perform
computations on an input by moving through a series of states or
configurations. A finite state machine (FSM) is an automaton in which
the state set contains only a finite number of
elements~\cite{basics-of-automata-theory}.

Two major types of finite state machines are the \emph{Mealy} machine
and \emph{Moore} machine. A Mealy machine determines its output as a function of its current state and input, while a Moore machine's output is based purely on the current state.
\subsubsection{FSMs as Co-routines}
\label{sec:fsms_as_coroutines}
\TODO{Formally define co-routines}
By definition, co-routines preserve the state of the function between calls,
making them a viable programming model for finite state machines.

\subsubsection{Hardware Description Languages}
\subsubsection{High-level Synthesis}

\section{Related Work}

\subsection{greenery}
\emph{greenery}~\cite{greenery-github} is a set of tools for parsing
and manipulating regular expressions (greenery.lego), for producing
finite-state machines (greenery.fsm), and for freely converting between
the two.  The main feature of greenery is the ability to convert two
regexes into finite state machines, computing the intersection of the
two FSMs as a third FSM, and converting the third FSM back to a regex.
\TODO{This library might be useful if we can find patterns where the intersection of FSMs would allow us to reuse hardware, seems like a stretch goal though}

\section{Language Specification}
\label{sec:language_specification}
The language was designed to closely match the syntax of the host language (C)
to reduce the cost of learning and the burden of context switching between
languages.  Our only extension of the C-language is the yield keyword to
indicate the end of a clock cycle. We modify the semantics of certain C
operators to better suit hardware construction.  Finally, we restrict the use
of valid C syntax that is not necessary for the functionality of our language. 

\subsection{Grammar}
\begin{verbatim}
<stmt>   ::= <expr>  # expr
          |  <sym> <sym>  # variable declaration
          |  <sym> = <expr>  # assignment
          |  <sym> <sym> = <expr>  # declaration with assignment
          |  <sym> <sym>(<sym>, ..., <sym>) { <stmt>* } # function declaration
          |  if (<expr>) { <stmt>* }  # if then
          |  if (<expr>) { <stmt>* } else { <stmt>* }  # if then else
          |  while (<expr>) { <stmt>* }  # while loop
          |  for (<expr>; <expr>; <expr>) { <stmt>*}  # for loop
<expr>   ::= null
          |  <number>
          |  <sym>
          |  (<expr>)
          |  <expresson> <binop> <expr>
          |  <unop> <expr>
          |  <sym>(<expr>, ..., <expr>)  # function call
          |  yield
          |  return <expr>
          # TODO: Do we need something like reg[2:0]
<sym>    ::= /[a-zA-Z_][a-zA-Z_0-9]*/
<number> ::= /[0-9]+/
<binop>  ::= !, &, |, ~&, ~|, ^, ~^, +, -, *, /, %, <<, >>, >, >=, <,
          |  <=, ==, !=, &&, ||
<unop>   ::= ~, +, -, (<sym>)
\end{verbatim}
\subsection{Types}

\subsubsection{input}
The \texttt{input} type is used to describe inputs to the state machine.  The
value of an input variable is read-only.
\subsubsection{output}
The \texttt{output} type is used to describe outputs to the state machine. The
must be set during the first clock cycle (cannot be undefined) \TODO{Should we
instead initialize it to some value?}. An output type must only be written once
during a single clock cycle.
\subsubsection{reg}
The \texttt{reg} type is used to describe a register for implementing
sequential logic. An instance of \texttt{reg} will retain it's value till the
next clock cycle.  An instance of \texttt{reg} can be read from freely, but
only written to once during a clock cycle.
\subsubsection{wire}
The \texttt{wire} type is used for implementing combinational logic. They can
be read freely but only assigned once per clock cycle. They must be driven by a
continuous assign statement or from a port of a module.

\subsection{yield}

\subsection{Example Programs}
\subsubsection{Downsample module}
\begin{minted}{c}
void downsample(input A, output B, output valid) {
  for (int y = 0; y < SIZE_Y; ++y) {
    for (int x = 0; x < SIZE_X; ++x) {
      valid = (x % 2);
      B = A;
      yield;
    }
  }
}
// TODO: Need to decide on how this would be wrapped
// example:
//  setup A, B, valid
//  while (true) {
//    downsample(A, B, valid)
//  }
\end{minted}

\subsubsection{VGA Controller}
\begin{minted}{c}
bool column_in_display(x) {
  return x >= VGA_X_MIN && x < VGA_X_MAX;
}
bool row_in_display(y) {
  return y >= VGA_Y_MIN && y < VGA_Y_MAX;
}
void vga_controller(input A, output B, output valid) {
  for (int y = 0; y < size_y; y++) {
    for (int x = 0; x < size_x; x++) {
      valid = column_in_display(x) && row_in_display(y);
      B = A;
      yield;
    }
  }
}
\end{minted}


\section{Language Internals (Compiler)}
\label{sec:compiler}

\subsection{Parser}
Option 1: Use a tool like Bison to generate a parser from the above grammar. 

Option 2: Create something better than Bison (lightweight, intuitive, simple, verifiable?)

\subsection{Internal Representation of State Machines}
The next phase begins by converting the programs AST into a control flow graph,
giving us all the possible paths that the program can take during execution.
We can then observe all the paths between yield statements as possible state
transitions during a clock-cycle. With these paths we can also perform type
checking (i.e. ensure that the outputs are only written once in a clock cycle).

\subsection{Lowering to CoreIR}

\section{Milestones}
\begin{itemize}
	\item Proposal
	\item Parser
	\item Control flow graph analysis \TODO{Is this what you call constructing the CFG?}
	\item Infer transition set (paths between yields)
	\item Infer FSM state (function state)
	\item Implement a way to verify the internal representation of an FSM (programmatic interface or visual/graphical representation to check manually)
	\item Type check paths between yields
	\item Lower transitions and FSM state to CoreIR logic
\end{itemize}

\section{Conclusion}

\subsubsection*{Acknowledgments}

\section*{References}

\small
\bibliographystyle{abbrv}
\bibliography{proposal}

\end{document}
